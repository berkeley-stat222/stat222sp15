\documentclass[11pt, oneside]{article}   	% use "amsart" instead of "article" for AMSLaTeX format
\usepackage{geometry}                		% See geometry.pdf to learn the layout options. There are lots.
\geometry{letterpaper}                   		% ... or a4paper or a5paper or ... 
%\geometry{landscape}                		% Activate for rotated page geometry
%\usepackage[parfill]{parskip}    		% Activate to begin paragraphs with an empty line rather than an indent
\usepackage{graphicx}				% Use pdf, png, jpg, or eps§ with pdflatex; use eps in DVI mode
								% TeX will automatically convert eps --> pdf in pdflatex		
\usepackage{amssymb}

%-----------------------------------------------------------------------------
% Special-purpose color definitions (dark enough to print OK in black and white)
\usepackage{color}
% A few colors to replace the defaults for certain link types
\definecolor{orange}{cmyk}{0,0.4,0.8,0.2}
\definecolor{darkorange}{rgb}{.71,0.21,0.01}
\definecolor{darkgreen}{rgb}{.12,.54,.11}
%-----------------------------------------------------------------------------
% The hyperref package gives us a pdf with properly built
% internal navigation ('pdf bookmarks' for the table of contents,
% internal cross-reference links, web links for URLs, etc.)
\usepackage{hyperref}
\hypersetup{pdftex, % needed for pdflatex
  breaklinks=true, % so long urls are correctly broken across lines
  colorlinks=true,
  urlcolor=blue,
  linkcolor=darkorange,
  citecolor=darkgreen,
}

\usepackage{booktabs}


\title{Stat 222 Project 5: Industry Partners}
\date{}							% Activate to display a given date or no date

\begin{document}
\maketitle

\section{Companies}

For the final project of the course, you will work with a problem from an industry partner. The two partner companies this year are \href{www.wise.io}{Wise.io} and \href{channelmeter.com}{Channelmeter}. Both companies will be in class on Wednesday, April 8 to tell you about what they do and the specific projects you will be working on.

\section{Group selection}

The group selection for this project will be a bit different from what we've done before. We'll have four groups of six students each. In addition, you can pair up with one other student that you definitely want to work with, and we'll make sure you're in a group together. By Thursday, April 9, each pair of students who want to work together should email us with your preferred company (or you can state no preference, if you like both equally). We'll then create the groups and try to accommodate your company preferences. Group assignments will be sent out by Friday, April 10.

\section{Interactions with your partner company}

The first thing your group will need to do is set up a preliminary meeting with your industry partner, to clarify the project objectives and get details about obtaining the data. This should take place during the week of April 13. Depending on your schedules, it's ok if not all group members attend this meeting. It is your responsibility to discuss with your industry partners how they would like you to communicate with them (email, phone, in person, etc.) and how often they'd like to be updated about your progress.

\section{Progress reports}
We will have three in-class progress reports, on April 20, April 27, and May 4. This is an opportunity for each group to get feedback from the class and instructors about what you've done so far and what your plans are. In particular, for the April 20th progress report, you MUST describe the data you are working with and your initial plan. Discuss specific milestones that you need to hit between now and the end of the semester to complete your project.

\section{Written reports}
Your grade for this project will be based on a final written report, due 5/17. You need to turn in a rough draft of this report by  5/10, and we will give you an initial grade (similar to past rubrics) and specific suggestions about how to improve your report by 5/13. The report should be 10-15 pages, single-spaced, and not counting any appendices. Include any code you write in an appendix, and as usual do not refer to specific objects or functions in your report. Your report should be divided into logical sections. This will vary depending on the application but a good top-level structure to start with is Abstract, Introduction, Data, Methods, Results, and Conclusions.

\section{Schedule}

\begin{tabular}{l|l|l|l}
Week & Monday class & Wednesday class & Notes\\
\hline
April 13 & Git setup for groups & Work day & Obtain data - meet with company\\
April 20 & Progress reports & Work day &\\
April 27 & Progress reports & Work day &\\
May 4 & Progress reports & No class & (RRR week) Rough drafts due 5/10\\
May 11th & No class & No class & (Finals week) Feedback on rough drafts by 5/13\\
& & & Final reports due 5/17
\end{tabular}


\end{document}  
