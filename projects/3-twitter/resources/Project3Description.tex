\documentclass[11pt, oneside]{article}   	% use "amsart" instead of "article" for AMSLaTeX format
\usepackage{geometry}                		% See geometry.pdf to learn the layout options. There are lots.
\geometry{letterpaper}                   		% ... or a4paper or a5paper or ... 
%\geometry{landscape}                		% Activate for rotated page geometry
%\usepackage[parfill]{parskip}    		% Activate to begin paragraphs with an empty line rather than an indent
\usepackage{graphicx}				% Use pdf, png, jpg, or eps§ with pdflatex; use eps in DVI mode
								% TeX will automatically convert eps --> pdf in pdflatex		
\usepackage{amssymb}

%-----------------------------------------------------------------------------
% Special-purpose color definitions (dark enough to print OK in black and white)
\usepackage{color}
% A few colors to replace the defaults for certain link types
\definecolor{orange}{cmyk}{0,0.4,0.8,0.2}
\definecolor{darkorange}{rgb}{.71,0.21,0.01}
\definecolor{darkgreen}{rgb}{.12,.54,.11}
%-----------------------------------------------------------------------------
% The hyperref package gives us a pdf with properly built
% internal navigation ('pdf bookmarks' for the table of contents,
% internal cross-reference links, web links for URLs, etc.)
\usepackage{hyperref}
\hypersetup{pdftex, % needed for pdflatex
  breaklinks=true, % so long urls are correctly broken across lines
  colorlinks=true,
  urlcolor=blue,
  linkcolor=darkorange,
  citecolor=darkgreen,
}

%SetFonts

%SetFonts


\title{Stat 222 Project 3: Twitter}
\date{}							% Activate to display a given date or no date

\begin{document}
\maketitle

\section{Data Description}

For this project, your primary data source will be Twitter.  And you will be
expected to work with the Twitter API using Python to access this data. On
Monday (2/23), we will be briefly review Python, introduce JSON (Javascript
Object Notation), and demonstrate how to interact with the Twitter API using
the \href{https://github.com/sixohsix/twitter}{Python Twitter Tools}.  However,
you will need to do outside reading to get up to speed with these
tools. 

Python Twitter Tools is one of several Python packages for interacting with
the Twitter API.  It is fairly minimal and is the package used in
chapter 1 and 9 of \emph{Mining the Social Web}.
\begin{itemize}
\item \href{https://rawgit.com/ptwobrussell/Mining-the-Social-Web-2nd-Edition/master/ipynb/html/Chapter%201%20-%20Mining%20Twitter.html}{Chapter 1: Mining Twitter: Exploring Trending Topics, Discovering What People Are Talking About, and More}
\item \href{https://rawgit.com/ptwobrussell/Mining-the-Social-Web-2nd-Edition/master/ipynb/html/Chapter%209%20-%20Twitter%20Cookbook.html}{Chapter 9: Twitter Cookbook}
\end{itemize}



\section{Your Assignment}

Each group is responsible for creating a presentation that visually answers a set of
questions about this data. {\em Which} questions you answer is up to you, but
think about telling a story. The story will be more interesting if the
questions you address are related to each other in some way.
Here are a few example topics:
\begin{itemize}
\item investigate the relation of breaking news on Twitter versus traditional
  news sources
\item compare stop words usage on Twitter versus NY Times
\item chart how the ratio of positive versus negative words used in tweets
  involving some event (or issue) change over time
\item relate tweets about a TV show/movie/book to their viewers/ticket sales/sales
  over time
\end{itemize}

\subsection*{Timeline}

\begin{itemize}
\item Monday (2/23) Begin Twitter project
\item Wednesday (2/25) Poster presentations for airline data
\item Monday (3/2) Introduction to text mining (e.g., NLTK)
\item Wednesday (3/4) Pecha Kucha
\item Monday (3/9)
\item Wednesday (3/11) Practice presentations
\item Monday (3/16) Final presentations
\end{itemize}

\section{Initial Guidance (to do for first data debrief)}

First, 

\section{Next Steps}

On Wednesday 

\newpage
\section{Presentation Details}

All presentations will be given in the \textbf{Pecha
Kucha}\footnote{\url{http://en.wikipedia.org/wiki/PechaKucha}} style.  Pecha
Kucha presentations have a very rigid format and have grown in popularity.  A
Pecha Kucha presentation consists of 20 slides that are automatically advanced
20 seconds (20x20).  The complete presentation lasts exactly 6 minutes and 40
seconds.

This is a very constrained format.  So you will need to carefully plan and
prepare your talk.  Each group will have 4 members and each member will
be responsible for presenting 5 slides.  Since the slides will automatically
advance, you will need to practice your talks before you present in class.

Note that your group will need to have an official in-class practice presentation
on Wednesday (3/11).  After your practice presentation, you will receive feedback,
which you should incorporate in your final presentation on Monday (3/16).

\subsection*{How to make slides}

There are many ways to create slides, but make sure that you are able to
save your slides as a PDF.  Here are some possibilities for you to explore:
\begin{itemize}
\item Beamer\\
 \url{http://web.mit.edu/rsi/www/pdfs/beamer-tutorial.pdf}
\item Pandoc\\
 \url{http://johnmacfarlane.net/pandoc/demo/example9/producing-slide-shows-with-pandoc}
\item Powerpoint
\item Keynote
\end{itemize}

\subsection*{How \textbf{NOT} to make slides}

\begin{itemize}
\item Tufte's \emph{PowerPoint Is Evil}\\
 \url{http://archive.wired.com/wired/archive/11.09/ppt2.html}
\item Norvig's \emph{Gettysburg Cemetery Dedication}\\
 \url{http://norvig.com/Gettysburg/sld001.htm}
\item Efron's \emph{Thirteen rules}\\
 \url{http://statweb.stanford.edu/~ckirby/brad/other/2013ThirteenRules.pdf}
\end{itemize}

\end{document}  
